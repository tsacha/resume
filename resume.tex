%%% Document
\documentclass[a4paper]{tsacha-resume}

\begin{document}

\fname{Sacha}
\lname{Trémoureux}
\fonction{Administrateur Systèmes Linux}
\mail{sacha@tremoureux.fr}
\phone{+33 (0)7 86 46 93 68}

\lastupdated

\cvhead

\begin{minipage}[t]{0.25\textwidth}

  \section{Études}

  \study{Licence pro. ASRALL}{Admin. systèmes et réseaux à base de
    logiciels libres}{2012-13}{Nancy}{Mention Bien}
  \study{DUT Informatique}{Informatique, culture scientifique,
    sociale, et humaine}{2010-12}{Nantes}{}

  \section{Liens}

  \cvlink{Blog}{https://s.tremoureux.fr/}
  \cvlink{GitHub}{https://github.com/tsacha/}
  \cvlink{LinkedIn}{https://fr.linkedin.com/in/tsacha/}


  \section{Compétences}
  \subsection{OS}
  RHEL et dérivées \\
  Debian et dérivées \\
  Exp. régulière sous Windows Server

  \subsection{Technologies}

  \subsubsection{Supervision}
  Icinga2, Prometheus, Nagios, Centreon

  \subsubsection{Industrialisation}
  Ansible, Puppet, SaltStack

  \subsubsection{Virtualisation}
  LXC, Docker, KVM

  \subsubsection{Outillage}
  GLPI, FusionInventory

  \subsubsection{Web}
  Apache, Nginx, Varnish, Squid

  \subsubsection{Mail}
  Doveot, Postfix, Zimbra

  \subsubsection{Sécurité}
  iptables, SELinux, Kerberos


  \subsection{Langages}
  \subsubsection{Utilisation régulière}
  Python\\
  Bash

  \subsubsection{Utilisation occasionnelle}
  Go \\
  Ruby, Perl, Powershell\\
  HTML / CSS / PHP

\end{minipage}
\hfill
\begin{minipage}[t]{0.67\textwidth}
  \section{Expérience}

  \company{Capensis}{Administrateur Systèmes}{Depuis Octobre 2013}{Nantes}
  \responsability{La Poste, DISIT}{Nantes (44)}{Depuis Juin 2016}
  \xp{Conception d'une architecture complète de collecte de données de
    supervision}{Icinga2, Docker, Ansible, RabbitMQ, Python3, ETCD…}
  \txp{Utilisation initiale d'Icinga2 puis migration par plusieurs itérations vers une solution développée en Python respectant :}
  \bulletxp{les contraintes de sécurité}
  \bulletxp{le besoin de scalabilité horizontale}
  \bulletxp{un nombre conséquent d’équipements supervisés en SNMP par
    minute}
  \bulletxp{la nécessité d'être déployée rapidement et facilement}

  \responsability{DCNS}{Brest (29)}{Avril 2015 à Décembre 2015}
  \xp{Évolution de Puppet déployé sur un parc de 2 000 serveurs hétérogènes}{Puppet, Foreman, r10k — Windows Server, AIX, Solaris, RedHat}
  \xp{Maintien et développement d'une infrastructure d'orchestration}{MCollective, ActiveMQ}
  \xp{Maintien de GLPI, création de plugins et scripts d'inventaire}{GLPI, FusionInventory}
  \xp{Gestion de Centreon, développement de sondes de supervision}{Centreon, Powershell, Perl}

  \responsability{Decathlon}{Lille (59)}{Octobre 2014 à Décembre 2014}
  \xp{Homogénéisation, industrialisation et supervision d'un parc de serveurs}{Puppet, Centreon, ActiveDirectory}

  \responsability{Missions au forfait de courte durée}{France}{}
  \xp{Déploiement industriel de solutions web (Canopsis, Talend)}{Puppet, Ansible}
  \xp{Installations et mises à niveau de solutions de
    supervision}{Icinga2, Nagios, Centreon}
  \xp{Développments réguliers de sondes de supervision}{SNMP, PHP, Perl, Ruby, Powershell}
  \bulletxp{infrastructures logicielles \textit{(HTTP, mail, AD…)}}
  \bulletxp{matérielles \textit{(ESX, 3Par, Tina, HP Blade…)}}
  \xp{Mise en place d'architectures web en
    haute-disponibilité}{Corosync, Pacemaker, KeepAlived, HAProxy, Nginx…}
  \xp{Dispense de formations professionnelles}{PostgreSQL, RH135, Centreon}

  \company{Findspire}{Stage | Administrateur Systèmes}{Juin 2013 à Octobre 2013}{Nancy}
  \xp{Industrialisation avec Puppet d'un environnement web en haute disponibilité pour un réseau social artistique}{Puppet, Git, Docker, LXC, Gunicorn}

  \section{Langues}
  \cvline{Anglais : bonne compréhension orale et écrite, écrit technique}

  \section{Centres d'intérêts}
  \cvline{Cinéma international (D. Lynch, F. Fellini, A. Farhadi…)}
  \cvline{Communauté du logiciel libre (cours et présentations au sein d'associations)}
  \cvline{Hébergement associatif — Terrain d'expérimentations
    technologique}

\end{minipage}

\end{document}
